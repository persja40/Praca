\chapter{Wprowadzenie}
\label{cha:wprowadzenie}

Teoria gier wielu osobom kojarzy się z opisem gier towarzyskich między dwojgiem graczy, lecz takie rozgrywki to rzadkość w naszym zróżnicowanym świecie, gdzie zwykle w grę ekonomiczną, społeczną czy polityczną angażuje się wiele uczestników. 
W niniejszej pracy weźmiemy na tapet jeden z jej filarów, czyli gry \textit{n}-osobowe. W tym typie gier ważnym elementem strategii jest odpowiedni wybór koalicjantów. Oczywiście nie będziemy w stanie uwzględnić wszystkich czynników mogących mieć wkład do gry, ale przeanalizujemy dwa równania ewolucyjne, które mogłyby sterować graczami. Zaczynajmy ...

%---------------------------------------------------------------------------

\section{Cele pracy}
\label{sec:celePracy}

Celem poniższej pracy jest przeprowadzenie symulacji koalicji mieszanych oraz weryfikacji przewidywanych wyników.


%---------------------------------------------------------------------------

\section{Zawartość pracy}
\label{sec:zawartoscPracy}

!!!TO BE DONE!!! \cite{Now06} \cite{Hof98} \cite{Str01} \cite{Qt} \cite{Tut} \cite{Sza}