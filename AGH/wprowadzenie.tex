\chapter{Wprowadzenie}
\label{cha:wprowadzenie}

Teoria gier wielu osobom kojarzy się z opisem gier towarzyskich między dwojgiem graczy, lecz takie rozgrywki to rzadkość w naszym zróżnicowanym świecie, gdzie zwykle w grę ekonomiczną, społeczną czy polityczną angażuje się wielu uczestników. W niniejszej pracy zostaną umówione gry wieloosobowe o niepełnej informacji. W tym typie gier ważnym elementem strategii jest odpowiedni wybór koalicjantów. Oczywiście nie jest możliwe aby uwzględnić wszystkie czynniki mogące mieć wpływ na grę, ale zostaną przeanalizowane dwa równania, modelujące grę ewolucyjną, które mogłyby sterować graczami oraz dla jednego z nich przeprowadzona będzie analiza stabilności. Modelami gry użytymi w niniejszej pracy będzie gra 3-osobowa oraz gra wieloosobowa, w której gracze będą ustawieni w okręgu. Symulacje partii w przypadku gier 3-osobowych będą obrazowane jako trajektorie w trójwymiarowej przestrzeni prawdopodobieństw, natomiast dla gier wieloosobowych jako funkcje prawdopodobieństwa od czasu dla poszczególnych graczy.
