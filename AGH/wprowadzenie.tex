\chapter{Wprowadzenie}
\label{cha:wprowadzenie}

Teoria gier wielu osobom kojarzy się z opisem gier towarzyskich między dwojgiem graczy, lecz takie rozgrywki to rzadkość w naszym zróżnicowanym świecie, gdzie zwykle w grę ekonomiczną, społeczną czy polityczną angażuje się wiele uczestników. 
W niniejszej pracy zostanie umówiony jeden z jej filarów, czyli gry wieloosobowe. W tym typie gier ważnym elementem strategii jest odpowiedni wybór koalicjantów. Oczywiście nie jest to możliwe aby uwzględnić wszystkie czynniki mogące mieć wkład do gry, ale zostaną przeanalizowane dwa równania ewolucyjne, które mogłyby sterować graczami oraz przeprowadzona będzie analiza ich stabilności. Modelami gry użytymi w niniejszej pracy będzie gra 3-osobowa oraz gra wieloosobowa, w której gracze będą ustawieni w okręgu. Symulację partii w przypadku gier 3-osobowych będą obrazowane jako trójwymiarowe funkcje prawdopodobieństwa, natomiast dla gier wieloosobowych jako funkcje prawdopodobieństwa od czasu dla pojedynczego gracza.
%---------------------------------------------------------------------------


!!!TEST CYTATÓW!!! \cite{Now06} \cite{Hof98} \cite{Str01} \cite{Qt} \cite{Tut} \cite{Sza} \cite{Fsmd}