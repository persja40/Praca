\chapter{Podsumowanie}
\label{cha:podsumowanie}

Celem pracy było przeprowadzenie symulacji  powstawania koalicji przez stosowanie strategii mieszanych. Oczekiwane wyniki na podstawie analizy matematycznej nie odpowiadały ściśle wynikom uzyskanym z symulacji. W metodach analitycznych zakłada się bowiem pełną informację graczy, podczas gdy w symulacji zakładano grę o niepełnej informacji. 

W celu zmniejszenia błędu szacowanego prawdopodobieństwa, można by uwzględniać do niego jedynie k-ostatnich gier. Spowodowałoby to trafne szacowanie prawdopodobieństwa podczas monotonicznej gry przeciwników w k partiach, czego skutkiem byłyby punkty stabilne na krawędziach po przeprowadzeniu na nich k partii. W niniejszej pracy szacowane prawdopodobieństwa nie mają możliwości dojścia do 0 lub 1, jeśli decyzje przeciwników nie były monotoniczne przez całą grę. Skutkiem tego jest przemieszczanie się funkcji po krawędziach sześcianu, co mogłoby ustać po k partiach jeśli tylko one uwzględniane byłyby w szacowaniu. Oczywiście szybkość poruszania się po krawędziach spada z czasem, lecz nigdy nie dojdzie do sytuacji w której spadnie do 0. Szacowanie prawdopodobieństw na podstawie tylko niektórych partii rodzi problem natury doboru k. 

Zgodnie z analizą funkcje dążyły do krawędzi sześcianu, omijając punkty niestabilne oraz krawędzie reprezentujące niemożliwe do zawiązania koalicje.

Symulacje gier N-osobowych pokazały dążenie graczy do trwałych koalicji. Wartą przeanalizowania byłaby sytuacja losowej zmiany prawdopodobieństw części graczy w stanie ustalonym. Mogłoby to z czasem doprowadzić ustalenia nowego stanu ustalonego.

Używana tu metoda zastosowania koncepcji strategii mieszanych do określania koalicji jest owocem dyskusji autora z promotorem tej pracy.
