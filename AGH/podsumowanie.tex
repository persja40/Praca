\chapter{Podsumowanie}
\label{cha:podsumowanie}

Celem pracy było przyprowadzenie symulacji koalicji mieszanych przy użyciu ewolucyjnej teorii gier. Oczekiwane wyniki na podstawie analizy matematycznej nie odpowiadały ściśle wynikom uzyskanym z symulacji. Powodem tego była analiza gry o pełnej informacji, gdzie symulacja zakładała grę o niepełnej informacji. W celu zmniejszenia błędu szacowanego prawdopodobieństwa, można by uwzględniać do niego jedynie k-ostatnich gier. Spowodowałoby to trafne szacowanie prawdopodobieństwa podczas monotonicznej gry przeciwników, czego skutkiem byłyby punkty stabilne na krawędziach po przeprowadzeniu na nich k gier. Rodzi to jednak problem natury doboru k. Zgodnie z analizą funkcje dążyły do krawędzi sześcianu, omijając punkty niestabilne oraz krawędzie reprezentujące niemożliwe do zawiązania koalicje.

Symulacje gier N-osobowych pokazały dążenie graczy do trwałych koalicji. Wartą przeanalizowania byłaby sytuacja losowej zmiany prawdopodobieństw części graczy w stanie ustalonym. Powinno to z czasem doprowadzić ustalenia nowego stanu ustalonego. 
