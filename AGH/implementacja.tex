\chapter{Implementacja symulacji}
\label{cha:implementacja}

W tym rozdziale chciałbym przedstawić technologie i narzędzia użyte do wykonania symulacji oraz sposoby ich uruchomienia.

%---------------------------------------------------------------------------

\section{Środowisko QT}
\label{sec:qt}
Zdecydowałem się wykorzystać QT Creator IDE z kilku powodów, które zamierzam zaraz rozwinąć. Najważniejszą cechą środowiska jest udostępnienie go na kilku rodzajach licencji. Osobiście użyłem licencji LGPL, która pozwoliła mi bez ponoszenia kosztów korzystać ze środowiska. Kolejnym ważnym elementem jest multiplatformowość pozwalająca w łatwy sposób przenosić kod program między systemami operacyjnymi, o ile nie zostały użyte biblioteki dostępne tylko na jeden z systemów. Kolejną z zalet jest łatwy i intuicyjny interfejs tworzenia graficznego interfejsu użytkownika, osoba mająca wcześniej styczność z chociażby biblioteką Swing Java'y nie powinna mieć problemu z zaadaptowaniem się do  formularza QT Creatora. Wykorzystywany jest model sygnałów i slotów, polegający na emitowaniu sygnału przez zdarzenie, który następnie trafia do podłączonego slotu. Jest to w stanie znacznie ułatwić komunikację między elementami. Używanie nowoczesnego języka C++ (ja używałem wersji 14) nie sprawia problemów, lecz powinniśmy być świadomi że przykładowe uruchomienie wątków w aplikacji powinno być zrobione przy użyciu klas i funkcji z biblioteki QT.

%-------------------------------------------------------------------------------------------------------------------------------------------------------------

\section{GLWidget}
\label{sec:glwidget}

%-------------------------------------------------------------------------------------------------------------------------------------------------------------

\section{Schemat programu}
\label{sec::schemat}

%-------------------------------------------------------------------------------------------------------------------------------------------------------------

\section{Rysowanie 3D}
\label{sec::3d}

%-------------------------------------------------------------------------------------------------------------------------------------------------------------

\section{Makefile}
\label{sec::makefile}
Symulując grę w okręgu postanowiłem rysować wykresy funkcji prawdopodobieństwa od numeru partii. Do tego celu uznałem, że najbardziej odpowiedni będzie plik \textit{Makefile}, który wykona kompilację, uruchomienie oraz narysowanie wykresu przy pomocy programu gnuplot. Aby uruchomić program należy podać argument: G - ilość partii do rozegrania oraz P - ilość graczy. Poniżej przykładowe polecenie do wykonania symulacji dla 100 partii rozegranych przez 20 zawodników.
\begin{verbatim}
make G=100 P=20
\end{verbatim}
