\documentclass[pdflatex,11pt]{aghdpl}
% \documentclass{aghdpl}               % przy kompilacji programem latex
% \documentclass[pdflatex,en]{aghdpl}  % praca w języku angielskim
\usepackage[polish]{babel}
\usepackage[utf8]{inputenc}

% dodatkowe pakiety
\usepackage{hyperref}
\usepackage{url}
\hypersetup{
    colorlinks,
    citecolor=black,
    filecolor=black,
    linkcolor=black,
    urlcolor=black
}
\usepackage{enumerate}
\usepackage{listings}
\lstloadlanguages{TeX}

%moje pakiety
%zdjecia
\usepackage{wrapfig}
\usepackage{graphicx}
\usepackage{subfig}

%math
\usepackage{amsmath}

%draw
\usepackage{tikz}

%kod
\usepackage{listings}

%testy sklady
\usepackage{lipsum}

\lstset{
  literate={ą}{{\k{a}}}1
           {ć}{{\'c}}1
           {ę}{{\k{e}}}1
           {ó}{{\'o}}1
           {ń}{{\'n}}1
           {ł}{{\l{}}}1
           {ś}{{\'s}}1
           {ź}{{\'z}}1
           {ż}{{\.z}}1
           {Ą}{{\k{A}}}1
           {Ć}{{\'C}}1
           {Ę}{{\k{E}}}1
           {Ó}{{\'O}}1
           {Ń}{{\'N}}1
           {Ł}{{\L{}}}1
           {Ś}{{\'S}}1
           {Ź}{{\'Z}}1
           {Ż}{{\.Z}}1
}

%---------------------------------------------------------------------------

\author{Ernest Jęczmionek}
\shortauthor{E. Jęczmionek}

\titlePL{Symulacje ewolucji koalicji mieszanych}
\titleEN{Simulations of evolution of mixed coalitions}

\shorttitlePL{Symulacje ewolucji koalicji mieszanych} % skrócona wersja tytułu jeśli jest bardzo długi
\shorttitleEN{Thesis in \LaTeX}

\thesistypePL{Praca inżynierska}
\thesistypeEN{Bachelor of Science Thesis}

\supervisorPL{prof. dr hab. Krzysztof Kułakowski}
\supervisorEN{Professor Krzysztof Kułakowski}

\date{2017}

\departmentPL{Katedra Inforatyki Stosowanej i Fizyki Komputerowej}
\departmentEN{Department of Applied Informatics and Computational Physics}

\facultyPL{Wydział Fizyki i Informatyki Stosowanej}
\facultyEN{Faculty of Physics and Applied Computer Science}

\acknowledgements{Serdecznie dziękuję \dots tu ciąg dalszych podziękowań np. dla promotora, żony, sąsiada itp.}



\setlength{\cftsecnumwidth}{10mm}

%---------------------------------------------------------------------------

\begin{document}

\titlepages

\tableofcontents
\clearpage

\chapter{Wprowadzenie}
\label{cha:wprowadzenie}

Teoria gier wielu osobom kojarzy się z opisem gier towarzyskich między dwojgiem graczy, lecz takie rozgrywki to rzadkość w naszym zróżnicowanym świecie, gdzie zwykle w grę ekonomiczną, społeczną czy polityczną angażuje się wiele uczestników. 
W niniejszej pracy weźmiemy na tapet jeden z jej filarów, czyli gry \textit{n}-osobowe. W tym typie gier ważnym elementem strategii jest odpowiedni wybór koalicjantów. Oczywiście nie będziemy w stanie uwzględnić wszystkich czynników mogących mieć wkład do gry, ale przeanalizujemy dwa równania ewolucyjne, które mogłyby sterować graczami. Zaczynajmy ...

%---------------------------------------------------------------------------

\section{Cele pracy}
\label{sec:celePracy}

Celem poniższej pracy jest przeprowadzenie symulacji koalicji mieszanych oraz weryfikacji przewidywanych wyników.


%---------------------------------------------------------------------------

\section{Zawartość pracy}
\label{sec:zawartoscPracy}

!!!TO BE DONE!!! \cite{Now06} \cite{Hof98} \cite{Str01} \cite{Qt} \cite{Tut} \cite{Sza}
\chapter{Opis teoretyczny}
\label{cha:opis_teor}

\section{Gra}
\label{sec:gra}
W niniejszej pracy będziemy skupiali się na grach wieloosobowych, których szczególnym przypadkiem jest gra 3-osobowa. Model gry 3-osobowej pokazuje, że decyzję jednego z zawodników mają bezpośredni wpływ na zachowanie sąsiadów. W modelu gry N-osobowej decyzje graczy będą wpływać nie tylko na najbliższych sąsiadów, ale pośrednio także na decyzję graczy niesąsiadujących.
Omawiane tutaj gry są grami o niepełnej informacji, w tej pracy oznacza to grę, w której nie wszyscy uczestnicy znają prawdopodobieństwo wyborów przeciwników. Prawdopodobieństwo gry przeciwników będzie szacowane na podstawie obserwacji historii ich zagrań.

\section{Model gry}
\label{sec:model}
Jak wcześniej było wspomniane będą potrzebne dwa modele gier. Pierwszym z nich będzie model gry 3-osobowej \ref{fig:model_niezal}. Niech gracze nazywają się odpowiednio $G_0$, $G_1$, $G_2$ (gracz zerowy, pierwszy, drugi). Każdy z nich posiada prawdopodobieństwo zagrania, mającego na celu nawiązanie koalicji z graczem o wyższym indeksie. To prawdopodobieństwo oznaczone jest jako $p_i$ ($i$ jest indeksem gracza), przy czym dla $G_2$ gracz o wyższym indeksie to $G_0$. Prawdopodobieństwo zagrania, mającego na celu nawiązanie koalicji z graczem o niższym indeksie to $1 - p_i$, co nie będzie przechowywane z powodu że można w łatwy sposób to wyliczyć. Analogicznie jak wcześniej dla $G_0$ gracz o niższym indeksie to $G_2$. Ze względu że omawiana jest gra o niepełnej informacji żaden z zawodników nie ma dostępu do prawdopodobieństw innych graczy, lecz każdy ma dostęp do statystyki gry na którą składa się:
\begin{itemize}
\item $liczba_{partii}$ liczba rozegranych w grze partii
\item $nast_i$ ilość zagrań $G_i$ w celu zawiązania sojuszu z $G_{i+1}$, ilość zagrań $G_{i}$ aby nawiązać sojusz z $G_{i-1}$ wyraża się jako $liczba_{partii} - nast_i$
\end{itemize}
Model gry N-osobowe \ref{fig:model_zal} będzie dysponował takimi samymi danymi jak model gry 3-osobowej. Różnicą między tymi modelami będzie ustawienie graczy w okręgu, co będzie prowadziło do konsekwencji decyzji nieograniczających się do najbliższych sąsiadów.
{\color{red} UKŁAD}
\begin{figure}
	\centering
	\begin{tabular}{c|c}
	\subfloat[3-osobowej \label{fig:model_niezal}]{ 
		\begin{tikzpicture}[scale=0.2]
		\tikzstyle{every node}+=[inner sep=0pt]
		\draw [black] (40.6,-18.2) circle (3);
		\draw (40.6,-18.2) node {$G_0$};
		\draw [black] (55,-34.7) circle (3);
		\draw (55,-34.7) node {$G_1$};
		\draw [black] (26.6,-34.7) circle (3);
		\draw (26.6,-34.7) node {$G_2$};
		\draw [black] (42.57,-20.46) -- (53.03,-32.44);
		\fill [black] (53.03,-32.44) -- (52.88,-31.51) -- (52.12,-32.17);
		\draw (47.26,-27.9) node [left] {$p_0$};
		\draw [black] (52,-34.7) -- (29.6,-34.7);
		\fill [black] (29.6,-34.7) -- (30.4,-35.2) -- (30.4,-34.2);
		\draw (40.8,-34.2) node [above] {$p_1$};
		\draw [black] (28.54,-32.41) -- (38.66,-20.49);
		\fill [black] (38.66,-20.49) -- (37.76,-20.77) -- (38.52,-21.42);
		\draw (34.15,-27.89) node [right] {$p_2$};
		\end{tikzpicture}
	} &
	\subfloat[N-osobowej \label{fig:model_zal}]{  
		\begin{tikzpicture}[scale=0.2]
		\tikzstyle{every node}+=[inner sep=0pt]
		\draw [black] (38.2,-44.1) circle (3);
		\draw (38.2,-44.1) node {$G_0$};
		\draw [black] (51.9,-35.1) circle (3);
		\draw (51.9,-35.1) node {$G_1$};
		\draw [black] (23.5,-36.1) circle (3);
		\draw (23.5,-36.1) node {$G_{N-1}$};
		\draw [black] (51.9,-20.2) circle (3);
		\draw (51.9,-20.2) node {$G_2$};
		\draw [black] (23.5,-19.6) circle (3);
		\draw (23.5,-19.6) node {$G_{N-2}$};
		\draw [black] (37.6,-11.5) circle (3);
		\draw (37.6,-11.5) node {$...$};
		\draw [black] (40.71,-42.45) -- (49.39,-36.75);
		\fill [black] (49.39,-36.75) -- (48.45,-36.77) -- (49,-37.6);
		\draw (46.6,-40.1) node [below] {$p_0$};
		\draw [black] (26.14,-37.53) -- (35.56,-42.67);
		\fill [black] (35.56,-42.67) -- (35.1,-41.84) -- (34.62,-42.72);
		\draw (28.91,-40.6) node [below] {$p_{N-1}$};
		\draw [black] (23.5,-22.6) -- (23.5,-33.1);
		\fill [black] (23.5,-33.1) -- (24,-32.3) -- (23,-32.3);
		\draw (23,-27.85) node [left] {$p_{N-2}$};
		\draw [black] (51.9,-32.1) -- (51.9,-23.2);
		\fill [black] (51.9,-23.2) -- (51.4,-24) -- (52.4,-24);
		\draw (52.4,-27.65) node [right] {$p_1$};
		\draw [black] (49.34,-18.64) -- (40.16,-13.06);
		\fill [black] (40.16,-13.06) -- (40.59,-13.9) -- (41.11,-13.05);
		\draw (46.3,-15.35) node [above] {$p_2$};
		\draw [black] (35,-12.99) -- (26.1,-18.11);
		\fill [black] (26.1,-18.11) -- (27.04,-18.14) -- (26.55,-17.27);
		\draw (28.6,-15.05) node [above] {$p_{N-3}$};
		\end{tikzpicture}
	}
	\end{tabular}
\caption{Modele gry}
\label{fig:modele}
\end{figure}
%---------------------------------------------------------------------------------------------------------------------------------------------------------------------

\section{Równania standardowe}
\label{sec:r_stand}
W tym typie równań celem będzie maksymalizacja zysku( zyskiem jest prawdopodobieństwo ) w czasie. Wyprowadzenie będzie przeprowadzone dla $G_0$, przy założeniu gry o pełnej informacji. Pozostałe dwa równania można wyprowadzić analogicznie. Niech $x=p_0$, $y=p_1$, $z=p_2$. Wypłata dana jest przez:
\begin{equation}\label{eq:wypl_stand}
W = \underbrace{x(1-y)}_{\text{wypłata sojuszu $G_0$ z $G_1$}} + \overbrace{(1-x)z}^{\text{wypłata sojuszu $G_0$ z $G_2$}} \\
\end{equation}
Chcemy znaleźć dynamikę, więc różniczkujemy wypłatę po $\delta x$, co daje nam dynamikę wypłaty dla $G_0$ od czasu.
\begin{align}\label{eq:dynam_stand}
\frac{\delta W_x}{\delta t} = 1-y-z \nonumber\\
\delta W_x = \delta t (1-y-z)
\end{align}
W dalszej części będzie stosowane oznaczenie $\alpha = \delta t$. Równanie wygląda obiecująco, gdyż człon $1 - y$ jest prawdopodobieństwem zagrania $G_1$ aby zawiązać sojusz z $G_0$. Podobnie $-z$ prowadzi do sojuszu $G_2$ z $G_0$, minus przy $z$ powoduje ubytek dla $x$, ponieważ prawdopodobieństwo zagrania $G_0$ w celu zawarcia sojuszu z $G_2$ jest dane przez $1-x$.

Z powodu że tematem jest gra o niepełnej informacji należy zmodyfikować równanie \ref{eq:dynam_stand}. Opis użytych parametrów można znaleźć w podrozdziale \ref{sec:model}.
\begin{equation} \label{eq:stand}
\Delta p_i = \alpha \cdot (1 - \frac{nast_{i+1}}{liczba_{partii}} - \frac{nast_{i-1}}{liczba_{partii}})
\end{equation}

%---------------------------------------------------------------------------------------------------------------------------------------------------------------------

\section{Równania replikatorów}
\label{sec:r_repli}
Model dynamiki replikatorów jest najbardziej znanym różniczkowym modelem teorii gier ewolucyjnych, przez co może stanowić dobry wybór do sterowania zachowaniem graczy. Tak jak w poprzednim przypadku przyjmijmy prawdopodobieństwa kolejno $x$, $y$ oraz $z$ pamiętając, że prawdopodobieństwa poza tym z pochodnej są szacowane. Podstawowy wzór równania wygląda następująco:
\begin{equation}
\dot{x} = x \cdot ( W - \overline{W})
\end{equation}
Gdzie $W$ jest średnią wypłatą dla strategii $x$ (dla nas prawdopodobieństwem gry na x), natomiast $\overline{W}$ jest średnią wypłatą co daje nam:
{\color{red} UKŁAD}
\begin{align*}
\dot{x} = x \cdot ( \overbrace{(1-y)}^{W} - \overbrace{(x(1-y) + (1-x)z)}^{\overline{W}}) \\
\Downarrow \\
\dot{x} = x \cdot (1-x) \cdot (1-y-z)
\end{align*}
Co dla gry 3-osobowej generuje równania:
\begin{align} \label{eq:repli}
\Delta p_0 = \alpha p_0 \cdot (1 - p_0) \cdot (1 - \frac{n_1}{liczba_{partii}} - \frac{n_2}{liczba_{partii}}) \nonumber \\
\Delta p_1 = \alpha p_1 \cdot (1 - p_1) \cdot (1 - \frac{n_2}{liczba_{partii}} - \frac{n_0}{liczba_{partii}}) \\
\Delta p_2 = \alpha p_2 \cdot (1 - p_2) \cdot (1 - \frac{n_0}{liczba_{partii}} - \frac{n_1}{liczba_{partii}}) \nonumber
\end{align} 

%---------------------------------------------------------------------------------------------------------------------------------------------------------------------

\section{Ograniczenie prawdopodobieństwa}
\label{sec:ograniczenie}
Wszystkie zmiany prawdopodobieństwa muszą przejść funkcję ograniczającą, więcej o niej znajdziemy w sekcji \ref{sec:ograniczenie}. Jak uzyskać $\Delta p_i$ dowiemy się w dwóch następnych podrozdziałach.
\begin{equation} \label{eq:ograniczenie}
p_i = ogr( p_i + \Delta p_i)
\end{equation}

Jak zauważyliśmy powyższe równania w łatwy sposób mogą wyjść poza przedział $<0,1>$. Aby temu zapobiec każda inkrementacja prawdopodobieństwa musi być obłożona funkcją ograniczającą. Każde nowo obliczone prawdopodobieństwo podawane jest jako parametr do funkcji $ogr$, a dopiero jej rezultat jest przypisywany poszczególnym prawdopodobieństwom graczom. Zdecydowałem się użyć następującej funkcji:
\begin{displaymath}
ogr(p_i) = \left\{
\begin{array}{ll}
1 & \text{jeżeli } p_i > 1 \\
p_i & \text{jeżeli } 1 \geq p_i \geq 0 \\
0 & \text{jeżeli } p_i < 0
\end{array} 
\right.
\end{displaymath}

Zapewne naszą uwagę przykuł także parametr $\alpha$ obecny w powyższych równaniach. W pierwszym z nich użycie go jest konieczne, gdyż w przeciwnym przypadku szacowanie prawdopodobieństwa innych doprowadziłoby do zawiązania trwałych koalicji już po pierwszej partii. N oznacza zagranie w stronę zawodnika z wyższym numerem, natomiast P z niższym. Zapętla to się dla pierwszego i ostatniego gracza. Zobaczmy przykład dla równania standardowego, prawdopodobieństwa początkowego $\frac{1}{2}$ i $\alpha = 1$ zakładając że:
\begin{align*}
Gracz_0 = N, Gracz_1 = P, Gracz_2 = N && Gracz_0 = N, Gracz_1 = P, Gracz_2 = P\\
\left\{
\begin{array}{ll}
\Delta p_0 = (1 - 0 - 1) =  0 & p_0=\frac{1}{2}\\
\Delta p_1 = (1 - 1 - 1) =  -1 & p_1= 0\\
\Delta p_2 = (1 - 1 - 0) =  0 & p_2=\frac{1}{2}\\
\end{array} 
\right. &&
\left\{
\begin{array}{ll}
\Delta p_0 = (1 - 0 - \frac{1}{2}) =  0 & p_0=\frac{3}{4}\\
\Delta p_1 = (1 - \frac{1}{2} - 1) =  -\frac{1}{2} & p_1= 0\\
\Delta p_2 = (1 - 0 - \frac{1}{2}) =  \frac{1}{2} & p_2=\frac{3}{4}\\
\end{array}
\right.
\end{align*}
Schemat po lewej przedstawia pierwszą partię, a po lewej drugą. Jak widzimy prowadzi to do bardzo szybkich zmian prawdopodobieństwa, co może praktycznie uniemożliwiać jakiekolwiek zmiany sojuszy. Wartym rozważenia jest czy w równaniach replikatorów potrzebny będzie jakiś parametr zmniejszający dynamikę skoro posiadają człon postaci $x(1-x)$. Weźmy założenia z poprzedniego przykładu. Przeanalizujmy przykład: {\color{red} UKŁAD}
\begin{align*}
Gracz_0 = N, Gracz_1 = P, Gracz_2 = N \\
\left\{
\begin{array}{ll}
\Delta p_0 = \frac{1}{2} \cdot (1 - \frac{1}{2}) \cdot (1 - 0 - 1) =  0 & p_0=\frac{1}{2}\\
\Delta p_1 = \frac{1}{2} \cdot (1 - \frac{1}{2}) \cdot (1 - 1 - 1) =  0 & p_1= \frac{1}{4}\\
\Delta p_2 = \frac{1}{2} \cdot (1 - \frac{1}{2}) \cdot (1 - 0 - 1) =  0 & p_2=\frac{1}{2}\\
\end{array} 
\right.
\\
Gracz_0 = N, Gracz_1 = P, Gracz_2 = P \\
\left\{
\begin{array}{ll}
\Delta p_0 = \frac{1}{2} \cdot (1 - \frac{1}{2}) \cdot (1 - 0 - \frac{1}{2}) = \frac{1}{8} & p_0=\frac{5}{8}\\
\Delta p_1 = \frac{1}{4} \cdot (1 - \frac{1}{4}) \cdot (1 - \frac{1}{2} - 1) = -\frac{1}{32} & p_1= \frac{7}{32}\\
\Delta p_2 = \frac{1}{2} \cdot (1 - \frac{1}{2}) \cdot (1 - 0 - \frac{1}{2}) = \frac{1}{8}  & p_2=\frac{5}{8}\\
\end{array}
\right.
\end{align*}
Jak widzimy najszybsza zmiana zachodzi dla pierwszego gracza, który traci 25\% zaufania do gracza o wyższym indeksie. W kolejnej partii nie widzimy już tak dużych zmian. Należy odpowiedzieć na pytanie czy jest to na tyle dużo, aby wprowadzać parametr mający spowalniać dynamikę zmian. Powinno się poruszyć dwie kwestie. Zgadzam się że dynamika prawdopodobieństwa jest dla mnie akceptowalna(czyli wynosi do 10\%), ale tylko dla prawdopodobieństw które oddalają się od środka przedziału, a grę rozpoczynamy właśnie w nim. Drugą sprawą jest porównanie obu równań. Porównanie wyników, w które rekurencyjnie wkładamy czynnik wymnażający zmianę nie jest najłatwiejszą rzeczą do opisania. Celem członu $x(1-x)$ w równaniu replikatorów jest rozwiązanie problemu supersilnych, szybko tworzących się koalicji.

Postarajmy się teraz skupić na szczególnych przypadkach, gdy w obu osobnych grach żaden z graczy nie współpracował. Użyjemy równań standardowych, prawdopodobieństwo początkowe $\frac{1}{2}$ oraz $\alpha = 1$.

\begin{align*}
Gracz_0 = N, Gracz_1 = N, Gracz_2 = N && Gracz_0 = P, Gracz_1 = P, Gracz_2 = P \\
\left\{
\begin{array}{ll}
\Delta p_0 = (1 - 1 - 1) =  -1 & p_0=0\\
\Delta p_1 = (1 - 1 - 1) =  -1 & p_1= 0\\
\Delta p_2 = (1 - 1 - 1) =  -1 & p_2=0\\
\end{array} 
\right. &&
\left\{
\begin{array}{ll}
\Delta p_0 = (1 - 0 - 0) =  1 & p_0= 1\\
\Delta p_1 = (1 - 0 - 0) =  1 & p_1= 1\\
\Delta p_2 = (1 - 0 - 0) =  1 & p_2= 1\\
\end{array}
\right.
\end{align*}
Przypadek ten pokaże nam skutki braku ograniczenia w prawdopodobieństwie. Wiemy że teraz każdy z graczy wykona ruch przeciwny do poprzedniego co da nam w obu grach:
\begin{align*}
\left\{
\begin{array}{l}
\Delta p_0 = (1 - \frac{1}{2} - \frac{1}{2}) =  0 \\
\Delta p_1 = (1 - \frac{1}{2} - \frac{1}{2}) =  0 \\
\Delta p_2 = (1 - \frac{1}{2} - \frac{1}{2}) =  0 \\
\end{array} 
\right.
\end{align*}
Brak wpływu na zmiany, gracze dokonują wyboru jak poprzednio.
\begin{align*}
Gracz_0 = P, Gracz_1 = P, Gracz_2 = P && Gracz_0 = N, Gracz_1 = N, Gracz_2 = N \\
\left\{
\begin{array}{ll}
\Delta p_0 = (1 - \frac{1}{3} - \frac{1}{3}) =  \frac{1}{3} & p_0= \frac{1}{3}\\
\Delta p_1 = (1 - \frac{1}{3} - \frac{1}{3}) =  \frac{1}{3} & p_1= \frac{1}{3}\\
\Delta p_2 = (1 - \frac{1}{3} - \frac{1}{3}) =  \frac{1}{3} & p_2= \frac{1}{3}\\
\end{array} 
\right. &&
\left\{
\begin{array}{ll}
\Delta p_0 = (1 - \frac{2}{3} - \frac{2}{3}) =  -\frac{1}{3} & p_0= \frac{2}{3}\\
\Delta p_1 = (1 - \frac{2}{3} - \frac{2}{3}) =  -\frac{1}{3} & p_1= \frac{2}{3}\\
\Delta p_2 = (1 - \frac{2}{3} - \frac{2}{3}) =  -\frac{1}{3} & p_2= \frac{2}{3}\\
\end{array}
\right.
\end{align*}
Nie wróciliśmy to punktu wyjścia, w którym wyjściowym prawdopodobieństwem chęci gry na gracza z większym indeksem jest $\frac{1}{3}$ dla gry przedstawionej po lewej stronie i $\frac{2}{3}$ dla gry po prawej. W pamięci każdego z graczy jest liczba rozegranych partii ze swoimi rywalami co będzie prowadziło do niekoniecznie oczywistych zachowań, gdy przeanalizujemy ścieżki gier o wyborach z większym prawdopodobieństwie.
\begin{align*}
Gracz_0 = P, Gracz_1 = P, Gracz_2 = P && Gracz_0 = N, Gracz_1 = N, Gracz_2 = N \\
\left\{
\begin{array}{ll}
\Delta p_0 = (1 - \frac{1}{4} - \frac{1}{4}) =  \frac{1}{2} & p_0= \frac{5}{6}\\
\Delta p_1 = (1 - \frac{1}{4} - \frac{1}{4}) =  \frac{1}{2} & p_1= \frac{5}{6}\\
\Delta p_2 = (1 - \frac{1}{4} - \frac{1}{4}) =  \frac{1}{2} & p_2= \frac{5}{6}\\
\end{array} 
\right. &&
\left\{
\begin{array}{ll}
\Delta p_0 = (1 - \frac{2}{5} - \frac{2}{5}) =  -\frac{1}{2} & p_0= \frac{1}{6}\\
\Delta p_1 = (1 - \frac{2}{5} - \frac{2}{5}) =  -\frac{1}{2} & p_1= \frac{1}{6}\\
\Delta p_2 = (1 - \frac{2}{5} - \frac{2}{5}) =  -\frac{1}{2} & p_2= \frac{1}{6}\\
\end{array}
\right.
\end{align*}
Widzimy zmianę zachowania graczy spowodowaną sumą częstości gier przeciwników. Sytuacja takiej fluktuacji będzie się powtarzać, lecz z czasem będziemy tracić naszą pulę graczy z powodu czynnika prawdopodobieństwa. Teoretycznie mając do dyspozycji nieskończoną populację graczy oraz nieograniczony czas, istniałby przypadek nieskończonej ,,sinusoidy'' o zwiększającym się okresie. Sytuacja taka nie jest dla nas pożądana, co stanowi dodatkowy argument za użyciem parametru $\alpha < 1$ skutecznie niwelującego wystąpienie takich sytuacji.

\section{Rozwiązanie stacjonarne równań standardowych}
\label{sec:stab_stand}
Do ustalenia punktów stałych potrzebujemy $\Delta p_i = 0$ oraz pomińmy współczynnik $\alpha$. Należy rozwiązać układ równań, gdzie przyjmijmy że kolejne prawdopodobieństwa $x$, $y$, $z$:
\begin{equation}
\left\{
\begin{array}{c}
1 - y - z = 0 \\
1 - x - z = 0 \\
1 - x - y = 0
\end{array}
\right. \Rightarrow p_0 = p_1 = p_2 = \frac{1}{2}
\end{equation}
Z czego wynika że gra startująca w punkcie $(\frac{1}{2},\frac{1}{2},\frac{1}{2})$ nie powinna z niego wyjść. Byłoby tak gdyby prawdopodobieństwa użyte w równaniu były faktycznymi prawdopodobieństwami $p_i$, są one natomiast jedynie obserwacją zachowania pozostałych graczy. Jak już wcześniej wspominaliśmy jest ono dane jako $\frac{n_j}{liczba_{partii}}$, dzięki czemu gra w ogóle się odbywa.
\section{Stabilność równań replikatorów}
\label{sec:stab_repl}
Aby wyznaczyć stabilność równań replikatorów należy najpierw znaleźć punkty stałe. Dla przejrzystości postanowiłem pominąć parametr $\alpha$, który nie wnosi nic do obliczeń, wykorzystuję realne prawdopodobieństwa jako szacowane oraz przyjmuję następujące oznaczenia:
{\color{red} UKŁAD}
\begin{align*}
\begin{array}{l}
\dot{x} = \Delta p_0 \\
\dot{y} = \Delta p_1 \\
\dot{z} = \Delta p_2
\end{array}
&&
\left\{
\begin{array}{l}
\dot{x} = 0 \\
\dot{y} = 0 \\
\dot{z} = 0 
\end{array}
\right.
\Rightarrow kombinacje
\begin{array}{ll}
(0,0,0)  & i=0 \\
(\frac{1}{2},\frac{1}{2},\frac{1}{2}) & i=1 \\
(1,1,1) & i=2 \\
(0,1,\xi) & i=3 \\ 
\end{array}
\text{dają punkty stałe} (x^*_i, y^*_i, z^*_i)\text{, gdzie }\xi \in <0,1>
\end{align*}
Weźmiemy na tapet 4 przypadki, które nie są symetryczne względem siebie, ale najpierw musimy policzyć pochodne cząstkowe.
\begin{align*}
\begin{array}{l}
\frac{\delta \dot{x}}{\delta x} = 1-y-z-2x+2xy+2xz\\
\frac{\delta \dot{x}}{\delta y} = x^2 - x\\
\frac{\delta \dot{x}}{\delta z} = x^2 - x\\
\end{array}
&&
\begin{array}{l}
\frac{\delta \dot{y}}{\delta x} = y^2 - y\\
\frac{\delta \dot{y}}{\delta y} = 1-x-z-2y+2xy+2yz\\
\frac{\delta \dot{y}}{\delta z} = y^2 - y\\
\end{array}
&&
\begin{array}{l}
\frac{\delta \dot{z}}{\delta x} = z^2 - z\\
\frac{\delta \dot{z}}{\delta y} = z^2 - z\\
\frac{\delta \dot{z}}{\delta z} = 1-x-y-2z+2xz+2yz\\
\end{array}
\end{align*}
Macierz Jacobiego i wzór na wartości własne{\color{red} ZRÓB PIONOWĄ KRESKĘ PRZY =}
\begin{align*}
J= \left(
\begin{array}{ccc}
\frac{\delta \dot{x}}{\delta x} & \frac{\delta \dot{x}}{\delta y} & \frac{\delta \dot{x}}{\delta z} \\
\frac{\delta \dot{y}}{\delta x} & \frac{\delta \dot{y}}{\delta y} & \frac{\delta \dot{y}}{\delta y} \\
\frac{\delta \dot{z}}{\delta x} & \frac{\delta \dot{z}}{\delta y} & \frac{\delta \dot{z}}{\delta x}
\end{array}
\right)_{
	\begin{array}{c}
		x=x^*_i\\
		y=y^*_i\\
		z=z^*_i\\	
	\end{array}	
} = J_i
&&
\left|
\begin{array}{ccc}
\frac{\delta \dot{x}}{\delta x}-\lambda & \frac{\delta \dot{x}}{\delta y} & \frac{\delta \dot{x}}{\delta z} \\
\frac{\delta \dot{y}}{\delta x} & \frac{\delta \dot{y}}{\delta y}-\lambda & \frac{\delta \dot{y}}{\delta z} \\
\frac{\delta \dot{z}}{\delta x} & \frac{\delta \dot{z}}{\delta y} & \frac{\delta \dot{z}}{\delta z}-\lambda
\end{array}
\right| = 0
\end{align*}
\begin{align*}
J_{i,\lambda} = J_i - \lambda I && 
\begin{array}{l}
\lambda_0 = 1\\
\lambda_1 \in \{-\frac{1}{2}, \frac{1}{4}\}\\
\lambda_2 = 1\\
\lambda_3 \in \{-z, z-1\}
\end{array}
&& \wedge &&
Re \lambda_i < 0
\end{align*}
Warunek na ujemną część rzeczywistą wartości własnej eliminuje nam wszystkie lambdy poza $\lambda_3$. Możemy powiedzieć, że:
\begin{equation}
\lambda_3 \text{ jest marginalnie stabilna} \iff \xi \in \{ 0, 1\}
\end{equation}
Jako że analiza matematyczna nie przyniosła nam odpowiedzi co do stabilności spróbujmy przeanalizować równanie $\dot{z}$. W pierwszym wierszu widzimy jak $x$ ani $y$ nie mogą nawiązać sojuszu i w tym przypadku punkt $(0,0,1)$ będzie punktem stabilności w którym $z$ będzie pewnym zwycięzcą. W przypadku gdy $x$ i $y$ mają przeciwne wartości występuje trwała koalicja co nie daje szansy na wejście w żaden sojusz.
\begin{wraptable}{rh}{0.5\textwidth}
    \centering
    \caption{Na krawędzi sześcianu}
\label{tab:krawedz_prawd}
\begin{tabular}{c|c|c|c}
x & y & 1-x-y & $\dot{z}$       \\ \hline 
0 & 0 & 1     & $z \cdot (1-z)$  \\
0 & 1 & 0     & 0                \\
1 & 0 & 0     & 0                \\
1 & 1 & -1    & $-z \cdot (1-z)$
\end{tabular}
\end{wraptable}
Ostatni z przypadków jest przeciwieństwem pierwszego, w którym znowu dwójka z graczy uporczywie trwa na jedynce, co jest silnym bodźcem do dążenia w kierunku zera i zawiązania koalicji. Oczywiście omawiany teraz przykład do osiągnięcia stabilności wymagałby braku zmian decyzji pozostałych graczy.
{\color{red} UKŁAD}


\chapter{Implementacja symulacji}
\label{cha:implementacja}

W tym rozdziale chciałbym przedstawić technologie i narzędzia użyte do wykonania symulacji oraz sposoby ich uruchomienia.

%---------------------------------------------------------------------------

\section{Środowisko QT}
\label{sec:qt}
Zdecydowałem się wykorzystać QT Creator IDE z kilku powodów, które zamierzam zaraz rozwinąć. Najważniejszą cechą środowiska jest udostępnienie go na kilku rodzajach licencji. Osobiście użyłem licencji LGPL, która pozwoliła mi bez ponoszenia kosztów korzystać ze środowiska. Kolejnym ważnym elementem jest multiplatformowość pozwalająca w łatwy sposób przenosić kod program między systemami operacyjnymi, o ile nie zostały użyte biblioteki dostępne tylko na jeden z systemów. Kolejną z zalet jest łatwy i intuicyjny interfejs tworzenia graficznego interfejsu użytkownika, osoba mająca wcześniej styczność z chociażby biblioteką Swing Java'y nie powinna mieć problemu z zaadaptowaniem się do  formularza QT Creatora. Wykorzystywany jest model sygnałów i slotów, polegający na emitowaniu sygnału przez zdarzenie, który następnie trafia do podłączonego slotu. Jest to w stanie znacznie ułatwić komunikację między elementami. Używanie nowoczesnego języka C++ (ja używałem wersji 14) nie sprawia problemów, lecz powinniśmy być świadomi że przykładowe uruchomienie wątków w aplikacji powinno być zrobione przy użyciu klas i funkcji z biblioteki QT.



\section{GLWidget}
\label{sec:glwidget}

\section{Schemat programu}
\label{sec::schemat}


\section{Rysowanie 3D}
\label{sec::3d}

\section{Makefile}
\label{sec::makefile}
Symulując grę w okręgu postanowiłem rysować wykresy funkcji prawdopodobieństwa od numeru partii. Do tego celu uznałem, że najbardziej odpowiedni będzie plik \textit{Makefile}, który wykona kompilację, uruchomienie oraz narysowanie wykresu przy pomocy programu gnuplot. Aby uruchomić program należy podać argument: G - ilość partii do rozegrania oraz P - ilość graczy. Poniżej przykładowe polecenie do wykonania symulacji dla 100 partii rozegranych przez 20 zawodników.
\begin{verbatim}
make G=100 P=20
\end{verbatim}

\chapter{Wyniki}
\label{cha:wyniki}

\section{Gry 3-osobowe}
\label{sec:N3nzal}

\paragraph{Równania standardowe}
\label{sec:r_stan}
\begin{wrapfigure}{rh}{0.5\textwidth}
    \centering
    \includegraphics[width=0.5\textwidth]{pict/wyniki/stand100_10.png}   
    \caption{Równania standardowe: 100 partii, 10 instancji}
	\label{fig:stand50_10} 
\end{wrapfigure}

Postaramy się teraz przeanalizować wyniki symulacji z użyciem równań standardowych. Z analizy teoretycznej równań spodziewamy się zawiązania koalicji pomiędzy dwójką z graczy, wynikiem tego jest dążenie funkcji do krawędzi sześcianu. Po osiągnięcia trwałej koalicji trzeci gracz bezskutecznie stara się grać na jednego z koalicjantów. Jest to widoczne poprzez poruszanie się funkcji po krawędzi sześcianu dążącej do jego wierzchołków. Widać że funkcje idące od centrum zachowują się w miarę stabilnie idąc w kierunku jakiejś krawędzi nie zmieniają monotoniczności. Istnieją niewielkie wahania wynikające z prawdopodobieństwa, ale nie mają one większego wpływu w dążeniu do jednej z krawędzi i zmianę ich decyzji. 

Tak naprawdę koalicje zawiązują się tuż po zaczęciu gry i dążą bezpośrednio do stanu ustalonego,  ponieważ na początku gry szacowane prawdopodobieństwo u innych graczy jest bardzo duże. Wynika to z tego szacowane prawdopodobieństwo w pierwszej grze wynosi 0\% albo 100\% co definitywnie wskazuje aby grać na jednego z zawodników. Było to powodem dla którego wprowadziłem współczynnik $\alpha$ równy 0.1, który ma za zadanie blokować sytuacje w których od razu po pierwszej grze jesteśmy w trwałej koalicji.

Jeśli chcielibyśmy uzyskać szybszą grę powinniśmy używać większych współczynników $\alpha$, co skutkować będzie mniejszą szansą na zerwanie koalicji. Jeśli natomiast chcielibyśmy żeby gra przebiegała wolniej moglibyśmy obniżyć współczynnik $\alpha$, co doprowadziłoby do większej liczby zmian partnerów.


%\lipsum[1-5]

%---------------------------------------------------------------------------------------------------------------------------------------------------------
\paragraph{Równania replikatorów}
\label{sec:r_repl}

\begin{figure}
	\centering
	\begin{tabular}{c|c}
		\centering
		\subfloat[50 partii \label{fig:repl100_10}]{\includegraphics[width=.45\textwidth]{pict/wyniki/repl100_10.png}} 
		&
		\subfloat[250 partii \label{fig:repl250_10}]{\includegraphics[width=.45\textwidth]{pict/wyniki/repl250_10.png}}
		\\ \hline
		\subfloat[1000 partii \label{fig:repl1000_10}]{\includegraphics[width=.45\textwidth]{pict/wyniki/repl1000_10.png}}
		&
		\subfloat[10000 partii \label{fig:repl10000_10}]{\includegraphics[width=.45\textwidth]{pict/wyniki/repl10000_10.png}}
	\end{tabular}
\caption{Równania replikatorów: 10 instancji}
\label{fig:repl_10}
\end{figure}

Od równania tego rodzaju spodziewamy się mniejszego wkładu, a jest to spowodowane tym że człon z własnym prawdopodobieństwem jest funkcja kwadratową $x(1-x)$ której maksimum ma wartości 0.5 i wynosi ono 0.25. Na krańcach dziedziny funkcji $<0,1>$ przyjmuje wartość 0 (przypomina wielomian węzłowy Lagrange'a). Co jest znacznym ograniczeniem dynamiki funkcji w stosunku do równań standardowych. Dobrym tego przykładem jest rysunek \ref{fig:repl50_10} na którym widać jak dla 50 gier prawdopodobieństwa zachowujemy się dużo wolniej w porównaniu z równaniami standardowymi co jest bezpośrednim wynikiem omawianego członu. Ograniczenie zmiany prawdopodobieństwa wprowadza możliwość zmiany koalicji, ponieważ prawdopodobieństwo przejścia między koalicjami wzrasta ze spadkiem maksymalnej pojedynczej jego zmiany. Widoczne na rysunku \ref{fig:repl50_10} kiedy to w początkowej fazie kilka koalicji rozwiązuje się i zmieniają partnerów, co nie było widoczne dla równań standardowych. Na rysunku \ref{fig:repl250_10} widzimy że tak samo jak dla równania standardowych i w tym przypadku funkcje prawdopodobieństwa dążą do krawędzi sześcianu, jednak dzieje się to znacznie wolniej gdyż człon równania z własnym prawdopodobieństwem dąży do 0. Dopiero dla 1000 gier zaczynamy obserwować sytuację podobną do 50 gier dla równań standardowych kiedy to funkcje prawdopodobieństwa dochodzą do krawędzi sześcianu i jeden z osamotniony graczy za wszelką cenę stara się grać tylko na jednego z partnerów aby wybić go ze stanu równowagi, co prowadzi do zmierzania do wierzchołków sześcianu, gdzie stan ustalony wygląda jako koalicja dwóch graczy z jednym wyobcowanym zawodnikiem, który jest zdecydowany na grę z jednym graczem lecz partner nie odwzajemnia jego chęci. Aż 10000 gier było wymagane aby prawdopodobieństwa doszły do wierzchołka gdzie mamy wcześniej opisaną sytuację.
%\lipsum[1-5]

%---------------------------------------------------------------------------------------------------------------------------------------------------------
\section{Gry N-osobowe}
\label{sec:N3zal}
\begin{wrapfigure}{rh}{0.5\textwidth}
    \centering
    \includegraphics[width=0.5\textwidth]{pict/wyniki/g500p20}   
    \caption{Gra w okręgu: 500 partii, 20 graczy}
	\label{fig:podst} 
\end{wrapfigure}

Chciałbym teraz przeanalizować wyniki jednej z symulacji \ref{fig:podst}. Jak już wcześniej zaobserwowaliśmy równania replikatorów dają dużo mniejszą dynamikę decyzji graczy.

Peleton graczy tworzy stabilne koalicje około 300 partii które nie są w stanie ulec zmianie. Pozostałe przypadki tworzą niestabilne koalicje, które zmieniają się w czasie. Najlepszym tego przykładem jest gracz który początkowo gra w kierunku swojego prawego sąsiada, a później zapewne przez jego niechęć po kilku fluktuacjach zaczyna drastycznie zmieniać partnera swojej gry - zaznaczonych kolorem fioletowym. Czynnik losowy graczy utrudnia grę tylko z jednym wybranym partnerem, dlatego jak widzimy podczas pierwszych 50 gier dochodzi do dużej liczby zmian zachowań graczy. Szczególnie widoczne jest to w pierwszych 50 partiach, gdzie gracze dopiero szacują zachowanie sąsiadów. W kolejnych 50 partiach gracze zachowują się coraz bardziej liniowo, gdyż błąd przewidywanego i realnego prawdopodobieństwa przeciwnika spada. Nie wchodzący w stałą koalicja mogą należeć do łańcucha graczy niezdecydowanych lub jednostek znajdujących się pomiędzy dwoma silnymi koalicjami które nie dają szansy na przyłączenie się do żadnej. Wartym zauważenia jest fakt że ostatnia zmiana monotoniczności funkcji zachodzi dopiero około 300 gry.

\begin{wrapfigure}{rh}{0.5\textwidth}
    \centering
    \includegraphics[width=0.5\textwidth]{pict/wyniki/g10000p200}   
    \caption{Gra w okręgu: 10000 partii, 200 graczy}
	\label{fig:niechciani} 
\end{wrapfigure}

Rozpatrzmy teraz dużo dłuższą grę w której zaangażowanych jest więcej graczy, co może pokazać nam przypadki szczególne. Na rysunku widzimy że większość zawodników osiąga stabilne koalicję przed grą 1000. Występuje grupka kilku graczy którzy pomimo tak dużej ilości gier nie byli w stanie zawiązać trwałych koalicji.

Mogło to wynikać z dwóch faktów. Po pierwsze mogli znaleźć się między graczami znajdującymi się w trwały w sojuszach którzy nie byli zainteresowani wchodzeniem w nowe. Drugą przyczyną może być nieznajomość prawdziwego prawdopodobieństwa podejmowania decyzji przez sąsiadów które jest tylko wartością znaną z rozgrywki różnica pomiędzy faktycznym prawdopodobieństwem gry sąsiada a tym co reprezentował. W rozgrywce może się znacząco różnić wpływając na mylną ocenę prawdopodobieństwa gry zawodników i mogących ustalić stabilnej koalicji. Rysunek \ref{fig:niechciani} pokazuje przypadek w którym dwóch silnych koalicjantów nie jest zainteresowanych wejście w sojusz z osamotnionym zawodnikiem między nimi, który jak widać z tabelki !!!REF!!! ZRÓB TABELKĘ !!! powoli próbuje dążyć do stałej koalicji z jednym z graczy.  


\chapter{Podsumowanie}
\label{cha:podsumowanie}

Celem pracy było przeprowadzenie symulacji  powstawania koalicji przez stosowanie strategii mieszanych. Oczekiwane wyniki na podstawie analizy matematycznej nie odpowiadały ściśle wynikom uzyskanym z symulacji. W metodach analitycznych zakłada się bowiem pełną informację graczy, podczas gdy w symulacji zakładano grę o niepełnej informacji. 

W celu zmniejszenia błędu szacowanego prawdopodobieństwa, można by uwzględniać do niego jedynie k-ostatnich gier. Spowodowałoby to trafne szacowanie prawdopodobieństwa podczas monotonicznej gry przeciwników w k partiach, czego skutkiem byłyby punkty stabilne na krawędziach po przeprowadzeniu na nich k partii. W niniejszej pracy szacowane prawdopodobieństwa nie mają możliwości dojścia do 0 lub 1, jeśli decyzje przeciwników nie były monotoniczne przez całą grę. Skutkiem tego jest przemieszczanie się funkcji po krawędziach sześcianu, co mogłoby ustać po k partiach jeśli tylko one uwzględniane byłyby w szacowaniu. Oczywiście szybkość poruszania się po krawędziach spada z czasem, lecz nigdy nie dojdzie do sytuacji w której spadnie do 0. Szacowanie prawdopodobieństw na podstawie tylko niektórych partii rodzi problem natury doboru wartości parametru k. 

Zgodnie z analizą funkcje dążyły do krawędzi sześcianu, omijając punkty niestabilne oraz krawędzie reprezentujące niemożliwe do zawiązania koalicje.

Symulacje gier N-osobowych pokazały dążenie graczy do trwałych koalicji. Wartą przeanalizowania byłaby sytuacja losowej zmiany prawdopodobieństw części graczy w stanie ustalonym. Mogłoby to z czasem doprowadzić ustalenia nowego stanu ustalonego.

Używana tu metoda zastosowania koncepcji strategii mieszanych do określania koalicji jest owocem dyskusji autora z promotorem tej pracy.



% itd.
% \appendix
% \include{dodatekA}
% \include{dodatekB}
% itd.


\bibliographystyle{alpha}
\bibliography{bibliografia}
%\begin{thebibliography}{1}
%
%\bibitem{Dil00}
%A.~Diller.
%\newblock {\em LaTeX wiersz po wierszu}.
%\newblock Wydawnictwo Helion, Gliwice, 2000.
%
%\bibitem{Lam92}
%L.~Lamport.
%\newblock {\em LaTeX system przygotowywania dokumentów}.
%\newblock Wydawnictwo Ariel, Krakow, 1992.
%
%\bibitem{Alvis2011}
%M.~Szpyrka.
%\newblock {\em {On Line Alvis Manual}}.
%\newblock AGH University of Science and Technology, 2011.cccccc
%\newblock \\\texttt{http://fm.ia.agh.edu.pl/alvis:manual}.
%
%\end{thebibliography}

\end{document}
